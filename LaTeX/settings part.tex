% !TEX TS-program = pdflatex
% !TEX encoding = UTF-8 Unicode

\documentclass[a4paper, 13pt]{article} % A4 형식 및 글자 크기


% layout/spacing related packages
\usepackage[margin=0.5in]{geometry}
\usepackage{indentfirst} \parindent=1em	% 들여쓰기
\usepackage{microtype}
\usepackage{kotex}	% 한글 사용 가능
\usepackage{graphicx}
\usepackage{setspace}
\onehalfspacing

\begin{document}
	\section{환경설정}
	다음은 환경설정에 대한 설명이다.	환경설정은 크게 \textbf{검색폴더 선택, 동영상 확장자, 이미지 확장자, 문서 확장자, 기타 확장자, 검사예외 폴더, 최종 사용일자 설정} 등으로 나뉘어져 있다.
	
	검색폴더 선택칸을 제외하고 확장자 추가/삭제 버튼을 통해 \textbf{확장자를 변경한 후 확인 버튼을 누르면 변경된 값이 저장}된다.
	
	\begin{figure}[h]
		\centering
		\includegraphics[width=0.7\textwidth]{Figures/Setting}
		\caption{환경설정}
		\label{fig:setting}
	\end{figure}

	\subsection{검색폴더 선택}
	모든 폴더 및 파일을 검색하는 것이 아닌 \textbf{일부분의 폴더에서 검색을 하고 싶을때 사용}한다. 하지만 다음 이미지와 같이 \textbf{Program Files 나 Windows 폴더}의 경우, 이후에 나올 \textbf{검색예외폴더 설정창에서 체크박스를 해제}하여야 검색이 가능하다.
	
	\begin{figure}[h]
		\centering
		\includegraphics[height=0.35\textheight]{Figures/searchfolder}
		\caption{검색폴더 선택}
		\label{fig:searchfolder}
	\end{figure}
	
	\subsection{Movie 확장자}
	검색을 통해 메인 화면에 있는 동영상 섹션에 나열할 \textbf{동영상 파일들의 확장자를 나열해 놓은 칸}이다.
	우측의 \textbf{추가 버튼으로 동영상 섹션 내에서 검색하고 싶은 확장자를 더할 수 있으며 삭제 버튼으로 검색에서 제외하고 싶은 확장자를 제거} 할 수 있다.
	
	\begin{figure}[h]
		\centering
		\includegraphics[width=0.4\textwidth]{Figures/Movie}
		\caption{Movie 확장자}
		\label{fig:movie}
	\end{figure}

	\subsection{Image 확장자}
	검색을 통해 메인 화면에 있는 이미지 섹션에 나열할 \textbf{이미지 파일들의 확장자를 나열해 놓은 칸}이다. 동영상 확장자 추가 버튼과 마찬가지로 우측의 \textbf{추가 버튼으로 이미지 섹션 내에서 검색하고 싶은 확장자를 더할 수 있으며, 삭제 버튼으로 검색에서 제외하고 싶은 확장자를 제거} 할 수 있다.
	
	\begin{figure}[h]
		\centering
		\includegraphics[width=0.4\textwidth]{Figures/Image}
		\caption{Image 확장자}
		\label{fig:image}
	\end{figure}
	
	\subsection{Doc 확장자}		
	검색을 통해 메인 화면에 있는 문서 섹션에 나열할 \textbf{문서 파일들의 확장자를 나열해 놓은 칸}이다. 동영상 확장자 추가 버튼과 마찬가지로 \textbf{우측의 추가 버튼으로 문서 섹션 내에서 검색하고 싶은 확장자를 더할 수 있으며, 삭제 버튼으로 검색에서 제외하고 싶은 확장자를 제거} 할 수 있다.
	
	추가적으로 문서 확장자의 대표적 예인 \textbf{.txt 확장자가 기본 확장자에 포함되어 있지 않기에 검색을 하기 위해선 추가 버튼을 이용하여 .txt 확장자를 추가}하여야 한다.
	
	\begin{figure}[h]
		\centering
		\includegraphics[width=0.4\textwidth]{Figures/Doc}
		\caption{Doc 확장자}
		\label{fig:Doc}
	\end{figure}

	\subsection{기타 확장자}
	검색을 통해 메인 화면에 있는 기타 섹션에 \textbf{검색하길 원하는 파일들의 확장자를 나열해 놓는 칸}이다. 동영상 확장자 추가 버튼과 마찬가지로 우측의 \textbf{추가 버튼으로 기타 섹션 내에서 검색하고 싶은 확장자를 더할 수 있으며, 삭제 버튼으로 검색에서 제외하고 싶은 확장자를 제거} 할 수 있다.
	
	추가적으로 \textbf{초기 기타 확장자 칸에는 아무런 확장자도 적혀 있지 않기에 사용자가 추가를 해주어야만} 관련 확장자 파일이 검색 된다.
	
	\begin{figure}[h]
		\centering
		\includegraphics[width=0.4\textwidth]{Figures/etc}
		\caption{기타 확장자}
		\label{fig:etc}
	\end{figure}

	\subsection{검사예외폴더}	
	검사예외폴더 칸은 \textbf{검색시 소요되는 시간을 줄이기 위하여} 윈도우 운영체제와 대부분의 프로그램들이 설치되는 \textbf{ Windows 폴더와 Program Files 폴더를 체크박스를 통해 검색에서 제외하거나 포함}시키는 것이 가능하다.
	
	검사예외폴더 칸에는 기본적으로 \textbf{Windows와 Program 폴더만이 지정 가능}하다.
	
	\begin{figure}[h]
		\centering
		\includegraphics[width=0.4\textwidth]{Figures/except}
		\caption{검사예외폴더}
		\label{fig:except}
	\end{figure}

	\subsection{최종 사용일자}	
	최종 사용일자 칸은 검색 시 나열할 파일이 \textbf{최종 사용일자 칸에 적힌 날짜 이전의 사용 파일만}을 출력하는 일종의 조건문이다.
	
	사용 날짜에 상관없이 모든 파일을 출력을 하고 싶을때는 최종 사용일자를 0 일전으로 적으면 된다.
	
	\begin{figure}[h]
		\centering
		\includegraphics[width=0.4\textwidth]{Figures/date}
		\caption{최종 사용일자}
		\label{fig:date}
	\end{figure}

	\subsection{기본값 \& 확인}
	기본값 버튼은 확장자를 추가하거나 삭제 등으로 \textbf{변경사항이 생겼을 경우 프로그램에 내장 된 변경 전의 초기값}으로 돌아가게 된다. 또한 기본값은 현재 변경이 불가능하다.
	
	확인 버튼의 경우 \textbf{현재까지 변경한 내용을 저장}하기 위한 버튼이다.
	
	\begin{figure}[h]
		\centering
		\includegraphics[width=0.4\textwidth]{Figures/button}
		\caption{기본값 \& 확인}
		\label{fig:button}
	\end{figure}
\end{document}